\documentclass[]{article}
\usepackage{lmodern}
\usepackage{amssymb,amsmath}
\usepackage{ifxetex,ifluatex}
\usepackage{fixltx2e} % provides \textsubscript
\ifnum 0\ifxetex 1\fi\ifluatex 1\fi=0 % if pdftex
  \usepackage[T1]{fontenc}
  \usepackage[utf8]{inputenc}
\else % if luatex or xelatex
  \ifxetex
    \usepackage{mathspec}
  \else
    \usepackage{fontspec}
  \fi
  \defaultfontfeatures{Ligatures=TeX,Scale=MatchLowercase}
\fi
% use upquote if available, for straight quotes in verbatim environments
\IfFileExists{upquote.sty}{\usepackage{upquote}}{}
% use microtype if available
\IfFileExists{microtype.sty}{%
\usepackage[]{microtype}
\UseMicrotypeSet[protrusion]{basicmath} % disable protrusion for tt fonts
}{}
\PassOptionsToPackage{hyphens}{url} % url is loaded by hyperref
\usepackage[unicode=true]{hyperref}
\hypersetup{
            pdftitle={Run PT analyses},
            pdfauthor={Jason},
            pdfborder={0 0 0},
            breaklinks=true}
\urlstyle{same}  % don't use monospace font for urls
\usepackage[margin=1in]{geometry}
\usepackage{color}
\usepackage{fancyvrb}
\newcommand{\VerbBar}{|}
\newcommand{\VERB}{\Verb[commandchars=\\\{\}]}
\DefineVerbatimEnvironment{Highlighting}{Verbatim}{commandchars=\\\{\}}
% Add ',fontsize=\small' for more characters per line
\usepackage{framed}
\definecolor{shadecolor}{RGB}{248,248,248}
\newenvironment{Shaded}{\begin{snugshade}}{\end{snugshade}}
\newcommand{\KeywordTok}[1]{\textcolor[rgb]{0.13,0.29,0.53}{\textbf{#1}}}
\newcommand{\DataTypeTok}[1]{\textcolor[rgb]{0.13,0.29,0.53}{#1}}
\newcommand{\DecValTok}[1]{\textcolor[rgb]{0.00,0.00,0.81}{#1}}
\newcommand{\BaseNTok}[1]{\textcolor[rgb]{0.00,0.00,0.81}{#1}}
\newcommand{\FloatTok}[1]{\textcolor[rgb]{0.00,0.00,0.81}{#1}}
\newcommand{\ConstantTok}[1]{\textcolor[rgb]{0.00,0.00,0.00}{#1}}
\newcommand{\CharTok}[1]{\textcolor[rgb]{0.31,0.60,0.02}{#1}}
\newcommand{\SpecialCharTok}[1]{\textcolor[rgb]{0.00,0.00,0.00}{#1}}
\newcommand{\StringTok}[1]{\textcolor[rgb]{0.31,0.60,0.02}{#1}}
\newcommand{\VerbatimStringTok}[1]{\textcolor[rgb]{0.31,0.60,0.02}{#1}}
\newcommand{\SpecialStringTok}[1]{\textcolor[rgb]{0.31,0.60,0.02}{#1}}
\newcommand{\ImportTok}[1]{#1}
\newcommand{\CommentTok}[1]{\textcolor[rgb]{0.56,0.35,0.01}{\textit{#1}}}
\newcommand{\DocumentationTok}[1]{\textcolor[rgb]{0.56,0.35,0.01}{\textbf{\textit{#1}}}}
\newcommand{\AnnotationTok}[1]{\textcolor[rgb]{0.56,0.35,0.01}{\textbf{\textit{#1}}}}
\newcommand{\CommentVarTok}[1]{\textcolor[rgb]{0.56,0.35,0.01}{\textbf{\textit{#1}}}}
\newcommand{\OtherTok}[1]{\textcolor[rgb]{0.56,0.35,0.01}{#1}}
\newcommand{\FunctionTok}[1]{\textcolor[rgb]{0.00,0.00,0.00}{#1}}
\newcommand{\VariableTok}[1]{\textcolor[rgb]{0.00,0.00,0.00}{#1}}
\newcommand{\ControlFlowTok}[1]{\textcolor[rgb]{0.13,0.29,0.53}{\textbf{#1}}}
\newcommand{\OperatorTok}[1]{\textcolor[rgb]{0.81,0.36,0.00}{\textbf{#1}}}
\newcommand{\BuiltInTok}[1]{#1}
\newcommand{\ExtensionTok}[1]{#1}
\newcommand{\PreprocessorTok}[1]{\textcolor[rgb]{0.56,0.35,0.01}{\textit{#1}}}
\newcommand{\AttributeTok}[1]{\textcolor[rgb]{0.77,0.63,0.00}{#1}}
\newcommand{\RegionMarkerTok}[1]{#1}
\newcommand{\InformationTok}[1]{\textcolor[rgb]{0.56,0.35,0.01}{\textbf{\textit{#1}}}}
\newcommand{\WarningTok}[1]{\textcolor[rgb]{0.56,0.35,0.01}{\textbf{\textit{#1}}}}
\newcommand{\AlertTok}[1]{\textcolor[rgb]{0.94,0.16,0.16}{#1}}
\newcommand{\ErrorTok}[1]{\textcolor[rgb]{0.64,0.00,0.00}{\textbf{#1}}}
\newcommand{\NormalTok}[1]{#1}
\usepackage{graphicx,grffile}
\makeatletter
\def\maxwidth{\ifdim\Gin@nat@width>\linewidth\linewidth\else\Gin@nat@width\fi}
\def\maxheight{\ifdim\Gin@nat@height>\textheight\textheight\else\Gin@nat@height\fi}
\makeatother
% Scale images if necessary, so that they will not overflow the page
% margins by default, and it is still possible to overwrite the defaults
% using explicit options in \includegraphics[width, height, ...]{}
\setkeys{Gin}{width=\maxwidth,height=\maxheight,keepaspectratio}
\IfFileExists{parskip.sty}{%
\usepackage{parskip}
}{% else
\setlength{\parindent}{0pt}
\setlength{\parskip}{6pt plus 2pt minus 1pt}
}
\setlength{\emergencystretch}{3em}  % prevent overfull lines
\providecommand{\tightlist}{%
  \setlength{\itemsep}{0pt}\setlength{\parskip}{0pt}}
\setcounter{secnumdepth}{0}
% Redefines (sub)paragraphs to behave more like sections
\ifx\paragraph\undefined\else
\let\oldparagraph\paragraph
\renewcommand{\paragraph}[1]{\oldparagraph{#1}\mbox{}}
\fi
\ifx\subparagraph\undefined\else
\let\oldsubparagraph\subparagraph
\renewcommand{\subparagraph}[1]{\oldsubparagraph{#1}\mbox{}}
\fi

% set default figure placement to htbp
\makeatletter
\def\fps@figure{htbp}
\makeatother


\title{Run PT analyses}
\author{Jason}
\date{!r Sys.Date()}

\begin{document}
\maketitle

\begin{Shaded}
\begin{Highlighting}[]
\KeywordTok{library}\NormalTok{(readxl) }\CommentTok{# This is used to import data from excel}
\KeywordTok{library}\NormalTok{(doBy)}
\KeywordTok{library}\NormalTok{(fuzzyjoin)}
\KeywordTok{library}\NormalTok{(reshape2)}
\KeywordTok{library}\NormalTok{(varhandle)}
\KeywordTok{library}\NormalTok{(multcomp)}
\end{Highlighting}
\end{Shaded}

\begin{verbatim}
## Loading required package: mvtnorm
\end{verbatim}

\begin{verbatim}
## Loading required package: survival
\end{verbatim}

\begin{verbatim}
## Loading required package: TH.data
\end{verbatim}

\begin{verbatim}
## Loading required package: MASS
\end{verbatim}

\begin{verbatim}
## 
## Attaching package: 'TH.data'
\end{verbatim}

\begin{verbatim}
## The following object is masked from 'package:MASS':
## 
##     geyser
\end{verbatim}

\begin{Shaded}
\begin{Highlighting}[]
\KeywordTok{library}\NormalTok{(car)}
\end{Highlighting}
\end{Shaded}

\begin{verbatim}
## Loading required package: carData
\end{verbatim}

\begin{Shaded}
\begin{Highlighting}[]
\KeywordTok{library}\NormalTok{(lattice)}
\KeywordTok{library}\NormalTok{(Formula)}
\KeywordTok{library}\NormalTok{(Hmisc)}
\end{Highlighting}
\end{Shaded}

\begin{verbatim}
## Loading required package: ggplot2
\end{verbatim}

\begin{verbatim}
## 
## Attaching package: 'Hmisc'
\end{verbatim}

\begin{verbatim}
## The following objects are masked from 'package:base':
## 
##     format.pval, units
\end{verbatim}

\begin{Shaded}
\begin{Highlighting}[]
\KeywordTok{library}\NormalTok{(ggpubr)}
\end{Highlighting}
\end{Shaded}

\begin{verbatim}
## Loading required package: magrittr
\end{verbatim}

\begin{Shaded}
\begin{Highlighting}[]
\KeywordTok{library}\NormalTok{(tidyverse)}
\end{Highlighting}
\end{Shaded}

\begin{verbatim}
## -- Attaching packages --------------------------------------------- tidyverse 1.3.0 --
\end{verbatim}

\begin{verbatim}
## v tibble  2.1.3     v dplyr   0.8.4
## v tidyr   1.0.2     v stringr 1.4.0
## v readr   1.3.1     v forcats 0.4.0
## v purrr   0.3.3
\end{verbatim}

\begin{verbatim}
## -- Conflicts ------------------------------------------------ tidyverse_conflicts() --
## x tidyr::extract()   masks magrittr::extract()
## x dplyr::filter()    masks stats::filter()
## x dplyr::lag()       masks stats::lag()
## x dplyr::recode()    masks car::recode()
## x dplyr::select()    masks MASS::select()
## x purrr::set_names() masks magrittr::set_names()
## x purrr::some()      masks car::some()
## x dplyr::src()       masks Hmisc::src()
## x dplyr::summarize() masks Hmisc::summarize()
\end{verbatim}

\begin{Shaded}
\begin{Highlighting}[]
\KeywordTok{library}\NormalTok{(here)}
\end{Highlighting}
\end{Shaded}

\begin{verbatim}
## here() starts at /Users/jcraggs/Documents/GitHub/Brad_PT
\end{verbatim}

\begin{Shaded}
\begin{Highlighting}[]
\KeywordTok{here}\NormalTok{()}
\end{Highlighting}
\end{Shaded}

\begin{verbatim}
## [1] "/Users/jcraggs/Documents/GitHub/Brad_PT"
\end{verbatim}

\begin{Shaded}
\begin{Highlighting}[]
\KeywordTok{dr_here}\NormalTok{(}\DataTypeTok{show_reason =} \OtherTok{FALSE}\NormalTok{)}
\end{Highlighting}
\end{Shaded}

\begin{verbatim}
## here() starts at /Users/jcraggs/Documents/GitHub/Brad_PT
\end{verbatim}

\begin{Shaded}
\begin{Highlighting}[]
\CommentTok{# library(ez) Not working for some reason March 17, 2020}
\CommentTok{# library(readr)        # ALREADY IN TIDYVERSE}
\CommentTok{# library(ggplot2)      # ALREADY IN TIDYVERSE}
\CommentTok{# library(dplyr)        # ALREADY IN TIDYVERSE}
\end{Highlighting}
\end{Shaded}

\begin{Shaded}
\begin{Highlighting}[]
\CommentTok{# import data:}
\CommentTok{#   This reads the files created by RedCap}
\KeywordTok{source}\NormalTok{(}\StringTok{'ImagingPsychometrics-BradPTConference_R_2020-03-16_1300.r'}\NormalTok{)}
\end{Highlighting}
\end{Shaded}

\begin{Shaded}
\begin{Highlighting}[]
\CommentTok{#       GROUPS}
\NormalTok{data }\OperatorTok\StringTok{ }\KeywordTok{count}\NormalTok{(group)}
\end{Highlighting}
\end{Shaded}

\begin{verbatim}
## # A tibble: 5 x 2
##   group          n
##   <labelled> <int>
## 1  0             1
## 2  1            53
## 3  2            80
## 4  3            52
## 5 NA             2
\end{verbatim}

\begin{Shaded}
\begin{Highlighting}[]
\NormalTok{dat1  <-}\StringTok{ }\KeywordTok{filter}\NormalTok{(data, group }\OperatorTok{!=}\StringTok{ }\DecValTok{0}\NormalTok{) }

\NormalTok{dat1 }\OperatorTok\StringTok{ }\KeywordTok{count}\NormalTok{(group)}
\end{Highlighting}
\end{Shaded}

\begin{verbatim}
## # A tibble: 3 x 2
##   group          n
##   <labelled> <int>
## 1 1             53
## 2 2             80
## 3 3             52
\end{verbatim}

\begin{Shaded}
\begin{Highlighting}[]
\NormalTok{dat1}\OperatorTok{$}\NormalTok{group.factor =}\StringTok{ }\KeywordTok{factor}\NormalTok{(dat1}\OperatorTok{$}\NormalTok{group,}\DataTypeTok{levels=}\KeywordTok{c}\NormalTok{(}\StringTok{"1"}\NormalTok{,}\StringTok{"2"}\NormalTok{,}\StringTok{"3"}\NormalTok{))}

\KeywordTok{levels}\NormalTok{(dat1}\OperatorTok{$}\NormalTok{group.factor)=}\KeywordTok{c}\NormalTok{(}\StringTok{"HC"}\NormalTok{,}\StringTok{"CLBP"}\NormalTok{,}\StringTok{"FM"}\NormalTok{, }\DataTypeTok{na.rm =} \OtherTok{FALSE}\NormalTok{)}

\KeywordTok{names}\NormalTok{(dat1)[}\KeywordTok{names}\NormalTok{(dat1) }\OperatorTok{==}\StringTok{ 'participant_id'}\NormalTok{] <-}\StringTok{ 'ID'}

\NormalTok{dat1 }\OperatorTok\StringTok{ }\KeywordTok{count}\NormalTok{(group.factor)}
\end{Highlighting}
\end{Shaded}

\begin{verbatim}
## # A tibble: 3 x 2
##   group.factor     n
##   <fct>        <int>
## 1 HC              53
## 2 CLBP            80
## 3 FM              52
\end{verbatim}

\begin{Shaded}
\begin{Highlighting}[]
\CommentTok{#       DONE AS OF March 17, 2020}
\end{Highlighting}
\end{Shaded}

\subsection{EXAMINE DATA}\label{examine-data}

\subsubsection{BDI}\label{bdi}

\begin{Shaded}
\begin{Highlighting}[]
\CommentTok{#       BDI VARIABLES}
\NormalTok{dat.bdi <-}\StringTok{ }\NormalTok{dat1[}\KeywordTok{c}\NormalTok{(}\DecValTok{1}\OperatorTok{:}\DecValTok{2}\NormalTok{,}\KeywordTok{grep}\NormalTok{(}\StringTok{"bdi"}\NormalTok{, }\KeywordTok{colnames}\NormalTok{(dat1)))]}
\KeywordTok{view}\NormalTok{(dat.bdi)}

\NormalTok{bdi.smry1  <-}\StringTok{ }\NormalTok{dat1 }\OperatorTok
\StringTok{    }\KeywordTok{group_by}\NormalTok{(group.factor) }\OperatorTok
\StringTok{    }\KeywordTok{summarise}\NormalTok{(}
        \DataTypeTok{count =} \KeywordTok{n}\NormalTok{(),}
        \DataTypeTok{avg.bdi.tot =} \KeywordTok{mean}\NormalTok{(bdi_total, }\DataTypeTok{na.rm =} \OtherTok{TRUE}\NormalTok{),}
        \DataTypeTok{sd.bdi.tot =} \KeywordTok{sd}\NormalTok{(bdi_total, }\DataTypeTok{na.rm =} \OtherTok{TRUE}\NormalTok{)}
        \CommentTok{# avg.pdi.tot = mean(bdi_total, na.rm = TRUE),}
        \CommentTok{# sd.pdi.tot = sd(pdi_total, na.rm = TRUE)}
\NormalTok{    )}
\KeywordTok{view}\NormalTok{(bdi.smry1)}
\end{Highlighting}
\end{Shaded}

\begin{Shaded}
\begin{Highlighting}[]
\CommentTok{#       BAR CHART}
\NormalTok{bdi.smry2  <-}\StringTok{ }\KeywordTok{ggplot}\NormalTok{(}\DataTypeTok{data =}\NormalTok{ dat1,}
                       \KeywordTok{aes}\NormalTok{(}\DataTypeTok{x =}\NormalTok{ group.factor,}
                        \DataTypeTok{y =}\NormalTok{ bdi_total,}
                        \DataTypeTok{fill =}\NormalTok{ group.factor)) }\OperatorTok{+}
\StringTok{    }\KeywordTok{geom_bar}\NormalTok{(}\DataTypeTok{color =} \StringTok{"black"}\NormalTok{, }\DataTypeTok{stat =} \StringTok{"identity"}\NormalTok{,}
             \DataTypeTok{position =} \KeywordTok{position_dodge}\NormalTok{(),}
             \DataTypeTok{size =}\NormalTok{ .}\DecValTok{3}\NormalTok{) }\OperatorTok{+}
\StringTok{    }\KeywordTok{scale_fill_hue}\NormalTok{(}\DataTypeTok{name =} \StringTok{"Groups"}\NormalTok{) }\OperatorTok{+}
\StringTok{    }\KeywordTok{xlab}\NormalTok{(}\StringTok{"Subjects groups"}\NormalTok{) }\OperatorTok{+}
\StringTok{    }\KeywordTok{ylab}\NormalTok{(}\StringTok{"BDI total"}\NormalTok{) }\OperatorTok{+}
\StringTok{    }\KeywordTok{ggtitle}\NormalTok{(}\StringTok{"BDI by group"}\NormalTok{)}

\KeywordTok{print}\NormalTok{(bdi.smry2) }\CommentTok{# I'm not sure what values are used in the graph}
\end{Highlighting}
\end{Shaded}

\begin{verbatim}
## Don't know how to automatically pick scale for object of type labelled/integer. Defaulting to continuous.
\end{verbatim}

\begin{verbatim}
## Warning: Removed 26 rows containing missing values (geom_bar).
\end{verbatim}

\includegraphics{Knit_PT_analyses_files/figure-latex/---\textgreater{} BDI Plots-1.pdf}

\begin{Shaded}
\begin{Highlighting}[]
\CommentTok{#       BOX PLOT}
\KeywordTok{ggplot}\NormalTok{(}\DataTypeTok{data =}\NormalTok{ dat1, }\DataTypeTok{mapping =} \KeywordTok{aes}\NormalTok{(}\DataTypeTok{x =}\NormalTok{ group.factor, }\DataTypeTok{y =}\NormalTok{ bdi_total)) }\OperatorTok{+}
\StringTok{    }\KeywordTok{geom_boxplot}\NormalTok{()}
\end{Highlighting}
\end{Shaded}

\begin{verbatim}
## Don't know how to automatically pick scale for object of type labelled/integer. Defaulting to continuous.
\end{verbatim}

\begin{verbatim}
## Warning: Removed 26 rows containing non-finite values (stat_boxplot).
\end{verbatim}

\includegraphics{Knit_PT_analyses_files/figure-latex/---\textgreater{} BDI Plots-2.pdf}

\begin{Shaded}
\begin{Highlighting}[]
\KeywordTok{ggboxplot}\NormalTok{(dat1, }\DataTypeTok{x =} \StringTok{"group"}\NormalTok{, }\DataTypeTok{y =} \StringTok{"bdi_total"}\NormalTok{,}
          \DataTypeTok{add =} \StringTok{"dotplot"}\NormalTok{)}
\end{Highlighting}
\end{Shaded}

\begin{verbatim}
## Don't know how to automatically pick scale for object of type labelled/integer. Defaulting to continuous.
\end{verbatim}

\begin{verbatim}
## Warning: Removed 26 rows containing non-finite values (stat_boxplot).
\end{verbatim}

\begin{verbatim}
## `stat_bindot()` using `bins = 30`. Pick better value with `binwidth`.
\end{verbatim}

\begin{verbatim}
## Warning: Removed 26 rows containing non-finite values (stat_bindot).
\end{verbatim}

\includegraphics{Knit_PT_analyses_files/figure-latex/---\textgreater{} BDI Plots-3.pdf}

\begin{Shaded}
\begin{Highlighting}[]
\KeywordTok{ggboxplot}\NormalTok{(dat1, }\DataTypeTok{x =} \StringTok{"group.factor"}\NormalTok{, }\DataTypeTok{y =} \StringTok{"bdi_total"}\NormalTok{,}
          \DataTypeTok{add =} \StringTok{"jitter"}\NormalTok{, }\DataTypeTok{shape =} \StringTok{"group.factor"}\NormalTok{)}
\end{Highlighting}
\end{Shaded}

\begin{verbatim}
## Don't know how to automatically pick scale for object of type labelled/integer. Defaulting to continuous.
\end{verbatim}

\begin{verbatim}
## Warning: Removed 26 rows containing non-finite values (stat_boxplot).
\end{verbatim}

\begin{verbatim}
## Warning: Removed 26 rows containing missing values (geom_point).
\end{verbatim}

\includegraphics{Knit_PT_analyses_files/figure-latex/---\textgreater{} BDI Plots-4.pdf}

\begin{Shaded}
\begin{Highlighting}[]
\KeywordTok{ggboxplot}\NormalTok{(dat1, }\DataTypeTok{x =} \StringTok{"group.factor"}\NormalTok{, }\DataTypeTok{y =} \StringTok{"bdi_total"}\NormalTok{,}
          \DataTypeTok{order =} \KeywordTok{c}\NormalTok{(}\StringTok{"FM"}\NormalTok{, }\StringTok{"CLBP"}\NormalTok{, }\StringTok{"HC"}\NormalTok{))}
\end{Highlighting}
\end{Shaded}

\begin{verbatim}
## Don't know how to automatically pick scale for object of type labelled/integer. Defaulting to continuous.
\end{verbatim}

\begin{verbatim}
## Warning: Removed 26 rows containing non-finite values (stat_boxplot).
\end{verbatim}

\includegraphics{Knit_PT_analyses_files/figure-latex/---\textgreater{} BDI Plots-5.pdf}

\begin{Shaded}
\begin{Highlighting}[]
\KeywordTok{ggboxplot}\NormalTok{(dat1, }\DataTypeTok{x =} \StringTok{"group.factor"}\NormalTok{, }\DataTypeTok{y =} \StringTok{"bdi_total"}\NormalTok{,}
          \DataTypeTok{color =} \StringTok{"red"}\NormalTok{, }\DataTypeTok{fill =} \StringTok{"darkblue"}\NormalTok{,}
          \DataTypeTok{order =} \KeywordTok{c}\NormalTok{(}\StringTok{"FM"}\NormalTok{, }\StringTok{"CLBP"}\NormalTok{, }\StringTok{"HC"}\NormalTok{))}
\end{Highlighting}
\end{Shaded}

\begin{verbatim}
## Don't know how to automatically pick scale for object of type labelled/integer. Defaulting to continuous.
\end{verbatim}

\begin{verbatim}
## Warning: Removed 26 rows containing non-finite values (stat_boxplot).
\end{verbatim}

\includegraphics{Knit_PT_analyses_files/figure-latex/---\textgreater{} BDI Plots-6.pdf}

\begin{Shaded}
\begin{Highlighting}[]
\CommentTok{# Mean plots}
\CommentTok{# ++++++++++++++++++++}
\CommentTok{# Plot weight by group}
\CommentTok{# Add error bars: mean_se}
\CommentTok{# (other values include: mean_sd, mean_ci, median_iqr, ....)}
\KeywordTok{ggline}\NormalTok{(dat1, }\DataTypeTok{x =} \StringTok{"group.factor"}\NormalTok{, }\DataTypeTok{y =} \StringTok{"bdi_total"}\NormalTok{, }
       \DataTypeTok{add =} \KeywordTok{c}\NormalTok{(}\StringTok{"mean_se"}\NormalTok{, }\StringTok{"jitter"}\NormalTok{), }
       \DataTypeTok{order =} \KeywordTok{c}\NormalTok{(}\StringTok{"FM"}\NormalTok{, }\StringTok{"CLBP"}\NormalTok{, }\StringTok{"HC"}\NormalTok{),}
       \DataTypeTok{ylab =} \StringTok{"BDI Total"}\NormalTok{, }\DataTypeTok{xlab =} \StringTok{"Group"}\NormalTok{)}
\end{Highlighting}
\end{Shaded}

\begin{verbatim}
## Warning in bind_rows_(x, .id): Vectorizing 'labelled' elements may not preserve
## their attributes
\end{verbatim}

\begin{verbatim}
## Warning in bind_rows_(x, .id): Vectorizing 'labelled' elements may not preserve
## their attributes
\end{verbatim}

\begin{verbatim}
## Don't know how to automatically pick scale for object of type labelled/integer. Defaulting to continuous.
\end{verbatim}

\begin{verbatim}
## Warning: Removed 26 rows containing non-finite values (stat_summary).
\end{verbatim}

\begin{verbatim}
## Warning: Removed 26 rows containing missing values (geom_point).
\end{verbatim}

\includegraphics{Knit_PT_analyses_files/figure-latex/---\textgreater{} BDI Plots-7.pdf}

\begin{Shaded}
\begin{Highlighting}[]
\CommentTok{# Box plots -   NOTY WORKING}
\CommentTok{# ++++++++++++++++++++}
\CommentTok{# Plot BDI by group and color by group}
\CommentTok{# bp <- ggboxplot(data = dat1, mapping = aes(x = group.factor, y = bdi_total, }
\CommentTok{#                                        fill = group.factor)) +}
\CommentTok{#   geom_boxplot()}

    
\CommentTok{#         color = "black", fill = }
\CommentTok{#         order = c("HC", "CLBP", "FM"),}
\CommentTok{#           ylab = "BDI", xlab = "Group")}

\CommentTok{# dat.totals <- dat1[c(1:2,grep("_total", colnames(dat1)))]}
\CommentTok{# view(dat.totals)}

\CommentTok{#       DONE AS OF March 17, 2020}
\end{Highlighting}
\end{Shaded}

\begin{Shaded}
\begin{Highlighting}[]
\CommentTok{#       COMPUTE ANOVA OF BDI }
\NormalTok{bdi.aov =}\StringTok{ }\KeywordTok{aov}\NormalTok{(bdi_total }\OperatorTok{~}\StringTok{ }\NormalTok{group.factor, }\DataTypeTok{data=}\NormalTok{dat1)}
\CommentTok{# print(bdi.aov)}

\CommentTok{#       SUMMARY OF THE ANALYSES}
\KeywordTok{summary}\NormalTok{(bdi.aov)}
\end{Highlighting}
\end{Shaded}

\begin{verbatim}
##               Df Sum Sq Mean Sq F value   Pr(>F)    
## group.factor   2   3909    1954   32.59 1.49e-12 ***
## Residuals    156   9354      60                     
## ---
## Signif. codes:  0 '***' 0.001 '**' 0.01 '*' 0.05 '.' 0.1 ' ' 1
## 26 observations deleted due to missingness
\end{verbatim}

\begin{Shaded}
\begin{Highlighting}[]
\KeywordTok{TukeyHSD}\NormalTok{(bdi.aov)}
\end{Highlighting}
\end{Shaded}

\begin{verbatim}
##   Tukey multiple comparisons of means
##     95% family-wise confidence level
## 
## Fit: aov(formula = bdi_total ~ group.factor, data = dat1)
## 
## $group.factor
##              diff      lwr       upr   p adj
## CLBP-HC  7.063077 3.616252 10.509902 8.9e-06
## FM-HC   12.853636 9.066032 16.641241 0.0e+00
## FM-CLBP  5.790559 2.213367  9.367752 5.4e-04
\end{verbatim}

\begin{Shaded}
\begin{Highlighting}[]
\KeywordTok{summary}\NormalTok{(}\KeywordTok{glht}\NormalTok{(bdi.aov, }\DataTypeTok{linfct =} \KeywordTok{mcp}\NormalTok{(}\DataTypeTok{group.factor =} \StringTok{"Tukey"}\NormalTok{)))}
\end{Highlighting}
\end{Shaded}

\begin{verbatim}
## 
##   Simultaneous Tests for General Linear Hypotheses
## 
## Multiple Comparisons of Means: Tukey Contrasts
## 
## 
## Fit: aov(formula = bdi_total ~ group.factor, data = dat1)
## 
## Linear Hypotheses:
##                Estimate Std. Error t value Pr(>|t|)    
## CLBP - HC == 0    7.063      1.457   4.849  < 1e-04 ***
## FM - HC == 0     12.854      1.601   8.030  < 1e-04 ***
## FM - CLBP == 0    5.791      1.512   3.830 0.000569 ***
## ---
## Signif. codes:  0 '***' 0.001 '**' 0.01 '*' 0.05 '.' 0.1 ' ' 1
## (Adjusted p values reported -- single-step method)
\end{verbatim}

\begin{Shaded}
\begin{Highlighting}[]
\KeywordTok{pairwise.t.test}\NormalTok{(dat1}\OperatorTok{$}\NormalTok{bdi_total, dat1}\OperatorTok{$}\NormalTok{group.factor,}
                \DataTypeTok{p.adjust.method =} \StringTok{"BH"}\NormalTok{)}
\end{Highlighting}
\end{Shaded}

\begin{verbatim}
## 
##  Pairwise comparisons using t tests with pooled SD 
## 
## data:  dat1$bdi_total and dat1$group.factor 
## 
##      HC      CLBP   
## CLBP 4.5e-06 -      
## FM   6.6e-13 0.00018
## 
## P value adjustment method: BH
\end{verbatim}

\begin{Shaded}
\begin{Highlighting}[]
\CommentTok{#   1. Homogeniety of variances}
\KeywordTok{plot}\NormalTok{(bdi.aov, }\DecValTok{1}\NormalTok{)}
\end{Highlighting}
\end{Shaded}

\includegraphics{Knit_PT_analyses_files/figure-latex/---\textgreater{} ANOVA BDI-1.pdf}

\begin{Shaded}
\begin{Highlighting}[]
\CommentTok{#   Levene's Test}
\KeywordTok{leveneTest}\NormalTok{(bdi_total }\OperatorTok{~}\StringTok{ }\NormalTok{group.factor, }\DataTypeTok{data =}\NormalTok{ dat1)}
\end{Highlighting}
\end{Shaded}

\begin{verbatim}
## Levene's Test for Homogeneity of Variance (center = median)
##        Df F value    Pr(>F)    
## group   2   11.86 1.604e-05 ***
##       156                      
## ---
## Signif. codes:  0 '***' 0.001 '**' 0.01 '*' 0.05 '.' 0.1 ' ' 1
\end{verbatim}

\begin{Shaded}
\begin{Highlighting}[]
\CommentTok{#   Relaxing the homogeneity of variance assumption}
\KeywordTok{oneway.test}\NormalTok{(bdi_total }\OperatorTok{~}\StringTok{ }\NormalTok{group.factor, }\DataTypeTok{data =}\NormalTok{ dat1)}
\end{Highlighting}
\end{Shaded}

\begin{verbatim}
## 
##  One-way analysis of means (not assuming equal variances)
## 
## data:  bdi_total and group.factor
## F = 45.125, num df = 2.000, denom df = 87.973, p-value = 3.259e-14
\end{verbatim}

\begin{Shaded}
\begin{Highlighting}[]
\KeywordTok{pairwise.t.test}\NormalTok{(dat1}\OperatorTok{$}\NormalTok{bdi_total, dat1}\OperatorTok{$}\NormalTok{group.factor,}
                \DataTypeTok{p.adjust.method =} \StringTok{"BH"}\NormalTok{,}
                \DataTypeTok{pool.sd =} \OtherTok{FALSE}\NormalTok{)}
\end{Highlighting}
\end{Shaded}

\begin{verbatim}
## 
##  Pairwise comparisons using t tests with non-pooled SD 
## 
## data:  dat1$bdi_total and dat1$group.factor 
## 
##      HC      CLBP  
## CLBP 1.3e-07 -     
## FM   3.3e-11 0.0015
## 
## P value adjustment method: BH
\end{verbatim}

\begin{Shaded}
\begin{Highlighting}[]
\CommentTok{#   2. Normality}
\CommentTok{#   plot(bdi.aov, 2) #  Produces errors}

\CommentTok{#   EXTRACT RESIDUALS}
\NormalTok{bdi.aov.resid <-}\StringTok{ }\KeywordTok{residuals}\NormalTok{(}\DataTypeTok{object =}\NormalTok{ bdi.aov)}

\CommentTok{#   RUN Shapiro-Wilk normality test}
\KeywordTok{shapiro.test}\NormalTok{(}\DataTypeTok{x =}\NormalTok{ bdi.aov.resid)}
\end{Highlighting}
\end{Shaded}

\begin{verbatim}
## 
##  Shapiro-Wilk normality test
## 
## data:  bdi.aov.resid
## W = 0.95426, p-value = 4.508e-05
\end{verbatim}

\begin{Shaded}
\begin{Highlighting}[]
\CommentTok{#   Non-parametric alternative to one-way ANOVA test}
\KeywordTok{kruskal.test}\NormalTok{(bdi_total }\OperatorTok{~}\StringTok{ }\NormalTok{group.factor, }\DataTypeTok{data =}\NormalTok{ dat1)}
\end{Highlighting}
\end{Shaded}

\begin{verbatim}
## 
##  Kruskal-Wallis rank sum test
## 
## data:  bdi_total by group.factor
## Kruskal-Wallis chi-squared = 54.056, df = 2, p-value = 1.828e-12
\end{verbatim}

\begin{Shaded}
\begin{Highlighting}[]
\CommentTok{#       DONE AS OF March 17, 2020}
\end{Highlighting}
\end{Shaded}

\end{document}
